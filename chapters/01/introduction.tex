\chapter{Introduction} \label{chap-introduction}
This book focuses on the fundamentals of Scientific Computing from two different perspectives;
Computer Science and Applied Mathematics. Initially, we provide information about Computer
Science related areas such as fundamentals of High-Performance Computing, \texttt{Linux}, 
programming languages and compilers. Then, we delve into mathematical modelling and scientific 
applications. Finally, we provide an analysis of the current advances in Cloud Computing, and 
explain its potential impact on \texttt{HPC} workloads.
\newline \newline 
Each chapter is devoted to a specific field and gives detailed explanation about its subject.
On the other hand, each of these subject are very broad to cover all of their aspects within a
chapter. Therefore, we highly suggest interested readers to refer to other references for a 
complete study. The main intention in this book is to explain how different components of an HPC 
focused system should be designed for efficient solutions to computationally demanding problems.
\newline \newline 
Current chapter contains a summary of each section. Picky readers can have a look at this part to
gather a general idea about a certain section if they don't want to read the whole book. However,
it is encouraged to follow the order in the book since chapters are connected to each other.
\newline \newline 
Chapter \ref{chap-hpc} begins with clarification of the fundamental concept in this book; 
High-Performance Computing which is also referred as \texttt{HPC}. A detailed explanation to the
questions like "What is HPC" or "Why/When is HPC needed" can be found in this section. This
chapter also introduces the "cluster" concept, and gives information about hardware and software
components of HPC clusters.  
\newline \newline 
Chapter \ref{chap-linux} is about operating systems, however it specifically focuses on 
\texttt{Linux}. This chapter starts with a historical background, then briefly explains the 
evolution of \texttt{Linux}. Additionally, some basic information about kernel is given in this 
chapter. Moreover, chapter continues with the shell environment which is probably the most important 
tool that an \texttt{HPC} user should be comfortable with.
\newline \newline 
Chapter \ref{chap-programming} is an introduction to the basic programming concepts
based on commonly used languages like \texttt{C}, \texttt{C++} and \texttt{Python}. 
\newline \newline 
Chapter \ref{chap-compilers} is devoted to how compilers work. Therefore, this chapter can be thought
as a bridge between the two concepts previously discussed; operating systems and programming. Our main
focus is on the GCC, however other compiler suites are discussed as well. Other than the compilers, 
we talk about debuggers and profilers too.
