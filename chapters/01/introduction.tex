\chapter{Introduction} \label{chap-introduction}
This book focuses on the fundamentals of Scientific Computing from two different perspectives;
Computer Science and Applied Mathematics. Initially, we provide information about Computer
Science related areas such as High-Performance Computing, Linux, programming languages
and compilers. Then, we delve into mathematical modelling and scientific applications. 
Finally, we provide an analysis of the current advances in Cloud Computing, and explain its 
potential impact on HPC workloads.
\newline \newline 
Each chapter is devoted to a specific field and gives detailed explanation about its subject.
However, interested readers should refer to other references for a complete study because main
intention in this book is to explain how different components of an HPC focused system should be
designed for efficient solutions.
\newline \newline 
Current chapter, chapter \ref{chap-introduction} summarizes each section. Thus, readers can have
a general idea by reading the descriptions in this section. Readers do not need to read
other parts of this book to understand a certain section, however the order of the chapters
are designed in a way that === WORK HERE!!!
\newline \newline 
Chapter \ref{chap-hpc} begins with clarification of the fundamental concept in this book; 
High-Performance Computing which is also referred as HPC. A detailed explanation to the
questions like "What is HPC" or "Why/When is HPC needed" can be found in this section. This
chapter also introduces the "cluster" concept, and gives information about hardware and software
components of HPC clusters.  
\newline \newline 
Chapter \ref{chap-linux} is about operating systems, however it specifically focuses on Linux.
This chapter starts with a historical background, then briefly explains the evolution of Linux.
Additionally, some basic information about kernel is given in this chapter. Moreover, chapter 
continues with the shell environment which is probably the most important tool that an HPC user
should be comfortable with.
\newline \newline 
Next chapter, chapter \ref{chap-programming} is an introduction to the basic programming concepts
based on commonly used languages like C, C++ and Python. 
\newline \newline 
Chapter \ref{chap-compilers} is devoted to how compilers work. Therefore, this chapter can be thought
as a bridge between the two concepts previously discussed; operating systems and programming. Our main
focus is on the GCC, however other compiler suites are discussed as well. Other than the compilers, 
we talk about debuggers and profilers too.
\newline \newline 
This book is completely written by the author and solely reflects his views.
