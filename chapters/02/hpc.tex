\chapter{High Performance Computing} \label{chap-hpc}
This chapter starts with the description of High-Performance Computing (HPC) and then 
provides information about various aspects of it. The final section is devoted to how 
\texttt{HPC} solutions are architected using crucial components of \texttt{HPC} environments.

\section{What is HPC?}
\texttt{HPC} is an ambiguous term unless an explicit definition is given. While an enterprise 
might use to represent a mainframe system that handles millions of financial transactions per 
second, a digital artist can refer to a workstation that renders gigabytes of videos. 
A general definition can be "A complex execution that is invoked to calculate a result" 
(REF HERE!!!). According to this definition, there are two key properties we can emphasize; 
\begin{itemize}
  \item Life-cycle: Unlike services or daemons, we expect \texttt{HPC} applications to end once 
computations are done.
  \item Complexity: \texttt{HPC} applications should handle complicated tasks that would normally 
be either impossible or at least be very hard to complete on a typical computer.
\end{itemize}
Based on the first item above a very basic example can be a small program that computes 
the area of a square from a given input. This code can take an input from the user, then 
execute the algorithm which is required to calculate the result. When it is complete, OS 
terminates it. 
\newline \newline
However if an \texttt{HPC} application becomes this simple, then we might have to refer all 
applications as \texttt{HPC} application which becomes odd. To narrow it down, we consider the 
second item above and assume that the calculation made by the computer should be based 
on an complicated algorithm such as solving turbulent wind flow over a hilly terrain, 
prediction of cavity bubbles around rotating ship propellers or accretion disks between 
compact astrophysical objects. 
\newline \newline
All these examples require implementation of complicated algorithms, but at the end what 
they do is getting an input and applying the algorithm required to produce a result. 
Once result is calculated, execution is terminated by the operating system on the platform 
they run. Such applications can be regarded as High Performance Computing applications.  

\section{What is Scientific HPC?}
Since examples above are scientific applications, we can also refer them as Scientific 
High Performance Computing applications. As a result, we avoid other HPC applications 
such as Video Analytics or Large DB query handling systems in our definition. 
Consequently our focus will be on the details of Scientific HPC in the rest of this book.

\section{HPC Clusters}
\subsection{Hardware: Fabrics, schedulers, storage etc.}
\subsection{Software: Management, security, modules etc.}
