\chapter{Linux Operating Systems and Shell} \label{chap-linux}
\texttt{Linux} is an operating system (OS) family. There are many different versions 
called distro, however they all use \texttt{Linux Kernel}. Therefore one can say that 
a \texttt{Linux} system is based on three components;
\begin{itemize}
  \item Kernel space: The part that talks to hardware and handles resource management.
  \item File system: File organization for efficient use of the system.
  \item User environment: All the other software and tools that is needed by users.
\end{itemize}
Linux kernel is developed by Linus Torvalds in 1991. He kept it open source and since 
then Linux became one of the largest and perhaps the most successful open source 
projects. Linus is still leading kernel development, however there are thousands of 
developers contributing it from all around the world. 

\section{Why is Linux Important?}
\texttt{Linux} systems are widely used by HPC centres. In fact as of August 2019 we can 
say that it is the only OS type currently being used by \texttt{Top-500} clusters. 
(REF HERE!!!). 
\newline \newline
Obviously this is not a coincidence, there are a few reasons for it:
\begin{itemize}
  \item \texttt{Linux} systems are free: Cost is always an issue.
  \item \texttt{Linux} systems are configurable: Users, groups and resources are all 
customizable.
  \item Most researchers are also working on \texttt{Linux}.
  \item \texttt{Linux} systems are more secure - because they're configurable.
\end{itemize}

\subsection{A Brief History of Operating Systems}
To better understand the importance of Linux, we need to know its evolution. Hence, we
should start with some historical background.
\newline \newline
In the late 60s, two computer scientists Dennis Ritchie and Ken Thompson developed the
first version of \texttt{Unix} at AT\&T Bell Laboratories. \texttt{Unix} was the first 
major multiuser and multitasking fully compatible OS. Its predecessor \texttt{MULTICS} 
(Multiplexed Information and Computing Service) was developed by GE \& MIT and AT\&T, 
but it was a failure. Thompson derived some ideas from \texttt{MULTICS} project and
merged it with the \texttt{C} language developed by Dennis Ritchie. As a result 
\texttt{Unix} was born. 
\newline \newline
70s were incredible productive in terms of OS development. Since AT\&T was unable to sell
\texttt{Unix} sell as a product due to the contract they made with government, they 
distributed to universities across the world. This lead many contributions to \texttt{Unix}.
When Ken Thompson was a visiting professor in California, he continued the development 
of \texttt{Unix} project and made significant extensions to it with his students. Hence,
\texttt{BSD} was born. 

\section{Linux Internals and Environment}
This book is not intended for explaining kernel details, however it is worth mentioning
some of the basics.


\section{Shell Scripting and Common Commands}
From a user perspective, perhaps the biggest advantage of using a Linux system would be
the flexibility provided by shell.
