\chapter{Compilers} \label{chap-compilers}
This section describes details about how a compiler works in a \texttt{Linux} 
environment. Our specific focus will be on the GNU Compiler Collection (GCC),
however general idea can be applied to other compilers as well.
\newline \newline
GCC is not the only compiler in the computer industry. There are lots of other
compilers as well. To name a few:
\begin{itemize}
  \item \texttt{CC}: C compiler. Developed by Denis Ritchie as a part of \texttt{Unix} project.
  It usually comes as the default compiler with \texttt{Linux} distributions. 
  \item \texttt{Visual Studio}: Microsoft's compiler suite for \texttt{Windows} platform.
  \item \texttt{Clang}: C and C++ compiler for Mac.
  \item \texttt{ICC}: Intel Compiler Suite.
  \item \texttt{XL}: IBM compilers. 
\end{itemize}
All these compilers and collection of compilers have pros and cons. Some of them 
specifically designed for certain platforms and cannot be used on hardware or 
software other than the ones that they are designed for. 
\newline \newline
For the purpose of this book \texttt{GCC} is the best option. Unlike \texttt{ICC}
or \texttt{XL}, it is not hardware specific. It supports almost every industry 
standard hardware, therefore readers are not restricted to certain hardware types.
Also, it is free to use. It is not completely open source like \texttt{Clang}, but
interested readers can refer to the \texttt{GitHub} repository for publicly 
available libraries. Most of the \texttt{Linux} distribution is shipped with a
\texttt{GCC} compiler, but it is always an option to compile and build \texttt{GCC}
from scratch.
\newline \newline
For these reasons, reader can always assume the compiler used in this book is
\texttt{GCC} unless otherwise stated explicitly.

\section{What is a Compiler?}
Following description is from the book "An Introduction to GCC" by Brian Gough;
\textit{Compiler is a software which converts a program from the textual source code, 
in a programming language such as C or C++, into machine code, the sequence 
of  1’s and 0’s used to control the central processing unit (CPU) of the computer}
(\cite{gough2004}, p. 7).
\newline \newline
In the same reference (\cite{gough2004}, p. 7), following source code is given:
\begin{lstlisting}[language=C]
    #include <stdio.h>
    int main (void)
    {
        printf ("Hello, world!\n");
        return 0; 
    }
\end{lstlisting}
This is a very simple \texttt{C} code, and its expected behaviour is printing 
"Hello, world!" to the screen. Let's assume that above source code is saved in a file 
called \texttt{hello.c}. If we run the command below in a terminal, this code gets
compiled and an output executable is generated. When we run this executable file, 
we get "Hello, world!" printed to the screen:
\begin{lstlisting}[language=bash]
    $ gcc -Wall hello.c -o hello
    $ ./hello
    Hello, world!
\end{lstlisting} 
By running the \texttt{gcc} command, we use \texttt{GCC}. Although \texttt{GCC} is a 
collection of compilers, \texttt{gcc} command can be used to compile codes written in 
\texttt{C} language. 
\newline
\newline
In the command above, we set the flag \texttt{-Wall} which is one of the many options of 
\texttt{gcc} and it means "print all warnings occurred during compilation". It should 
always be enabled for a cleaner code. Then we specify name of our source code and provide 
an output name after the \texttt{-o} option. Note that this is an optional flag. 
We could have used the following command as well:
\begin{lstlisting}[language=bash]
    $ gcc -Wall hello.c
\end{lstlisting} 
which spits out the following files:
\begin{lstlisting}[language=bash]
    $ ls -l
    -rw-r--r--  1 ertinaz  staff    87 28 Aug 16:20 hello.c
    -rwxr-xr-x  1 ertinaz  staff  8432 28 Aug 16:54 a.out
    $ ./a.out
    Hello, world!
\end{lstlisting} 
When an output name is not given compiler produces a file called \texttt{a.out} by 
default. It is same as the previous executable file, however providing an output name 
is much more convenient than the default naming.
\newline \newline
We will get into details about what goes on behind the scenes in the following
sections, but it is apparent that understanding the mechanism behind compilation
is a key factor for improving your skills to be able to use resources more efficiently..

\subsection{Why is it important to know how compilers work?}
Compiler is a very crucial aspect of your environment since it is the tool that converts
your source code to the binary executed by the machine. Therefore one can make a nicely 
written code perform even better using the correct compiler options. On the contrary 
choosing inconvenient compiler flags would result in a poorly performing output.

\section{How do compilers work?}
Compilers are sophisticated tools. On the surface there are five major processes when
one compiles a programme:
\begin{itemize}
  \item Pre-process: Merge all source code, e.g. include header files, add macros etc.
  \item Convert to assembly 
  \item Create object files
  \item Link shared libraries
  \item Generate binary executable
\end{itemize}
While this process depends on the type of compilation...

\subsection{Optimization using Compilers}
Compilers consist of three components:
\begin{itemize}
  \item Front-end: Modules that applies pre-processor.
  \item Back-end: Part that applies conversion from assembly to executable binary.
  \item Optimizer: Simply the performance enhancer. This is where the magic happens.
\end{itemize}
To prove my point please check out the following example. This source code is from
(\cite{gough2004}, p. 51):
\begin{lstlisting}[language=C]
    #include <stdio.h>
      
    double powern (double d, unsigned n)
    {
      double x = 1.0;
      unsigned j;

      for (j = 1; j <= n; j++)
        x *= d;
  
      return x; 
    }

    int main (void)
    {
      unsigned i;

      double sum = 0.0;
      for (i = 1; i <= 100000000; i++)
        sum += powern (i, i % 5);

      printf ("sum = %g\n", sum);

      return 0; 
    }
\end{lstlisting}
Details can be read in the reference. In short this code computes the $n-th$ power of a 
floating point number by repeated multiplication. Below you'll find execution times
of this code where two different optimization levels are used:
\begin{lstlisting}[language=bash, escapechar=!]
    $ gcc -Wall -O0 powern.c -lm -o powern-O0.x
    $ time ./powern-O0.x 
    sum = 4e+38

    real 0m0.869s
    !\colorbox{light-gray}{user 0m0.858s}!
    sys	 0m0.004s
\end{lstlisting}
In this attempt, we enforce no-optimization by using \texttt{-O0} flag. This is equivalent 
of not specifying a \texttt{-Olevel} option. When we execute the binary compiled this way, 
we complete execution in 0.858 seconds of CPU run-time. However, as you can see below just
changing optimization level from \texttt{-O0} to \texttt{-O3} and introducing a loop unrolling
based optimization, we can reduce run-time to 0.150 seconds which is almost 17.5\% of the 
previous run-time. 
\begin{lstlisting}[language=bash, escapechar=!]
    $ gcc -Wall -O3 -funroll-loops powern.c -lm \
        -o powern-O3.x
    $ time ./powern-O3.x 
    sum = 4e+38

    real 0m0.156s
    !\colorbox{light-gray}{user 0m0.150s}!
    sys	 0m0.003s
\end{lstlisting}

\section{Debugging, Profiling and Useful Linux Tools}
Explain GDB - Valgrind - GProf etc.
\newline \newline
Explain ldd - nm - strings etc.

\section{Tools for Automatic Compilation}
Explain Make - CMake - Bazel etc. here.
