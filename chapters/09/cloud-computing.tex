\chapter{Cloud Computing and HPC}
Even though cloud computing is not very new, one can argue that it has gained significant growth
in the last 5 years. Currently most of the vendors are providing \texttt{IaaS} and \texttt{PaaS}
solutions, and enterprises are transforming their business strategies to cloud native environments. 
\newline \newline
HPC and Scientific HPC are no exception to this situation.

\section{HPC as a Service}
Major cloud infrastucture providers offer HPC services as well.
\subsection{Amazon Web Services}
\subsection{IBM Cloud}
\subsection{Google Cloud}
\subsection{Microsoft Azure}
\subsection{OpenStack}
Unlike the previous vendors \texttt{OpenStack} is a Platform-as-a-Service (PaaS) offering. To simply
put, one can transform their HPC infrastructure to a cloud service using \texttt{OpenStack}.

\section{Virtulization and Containerization}
Virtual machines and containers are similar technologies desgined for similar purposes. The basic
idea behind their development is the abstraction of runtime environment from the host platform. 
For instance, one can create a virtual machine that uses \texttt{Linux} on top of a \texttt{Windows} 
machine and can run \texttt{Linux} application in that VM. This technology has been used very 
efficiently by data centers in the last 2 decades. However, there is one disadvantage of VMs; they
are a complete mimic of a host system and because of that they consume lots of resources.   
\newline \newline
Containerization solves this platform. Unlike VMs, containers are not a complete OS and therefore
they are much less light-weight compared to VMS.
\subsection{Docker Containers}
\subsection{Singularity}
\texttt{Singularity} is one of the most exciting projects in the HPC community. It is a containerization 
application which does not required priviliged user permissions. Therefore, it is more secure than
\texttt{Docker} and used widely by HPC centers.


\section{Orchestration and Scheduling}
Orchestration of the container deployments is a crucial task of the containerization management.
\subsection{Kubernetes}
\texttt{Kubernetes} became the de-facto standard of the container orchestrators.

\section{Storage and Networking as a Software}

\section{Performance and Comparison to Traditional HPC}
\subsection{InfiniBand and RDMA}

